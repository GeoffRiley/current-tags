\documentclass{article}
\usepackage{calc}
\usepackage[paperwidth=12.8in,paperheight=7.2in,margin=0.5in]{geometry}
\usepackage{microtype}
\usepackage{xifthen,eso-pic,multirow}

\usepackage{xcolor}
\definecolor{faded}{gray}{0.667}
\definecolor{full}{gray}{0.0}
\definecolor{none}{gray}{1.0}


\usepackage{ifpdf}
\ifpdf
   \usepackage[pdftex]{graphicx}
  % declare the path(s) where your graphic files are
   \graphicspath{{../img/}{./}{./img}}
   \DeclareGraphicsExtensions{.pdf,.png,.eps}
\else
   \usepackage[dvips]{graphicx}
  % declare the path(s) where your graphic files are
   \graphicspath{{../img}{./}{./img}}
   \DeclareGraphicsExtensions{.eps}
\fi


\usepackage{fontspec,unicode-math}
\defaultfontfeatures{Scale=MatchLowercase} 
\defaultfontfeatures{Ligatures=TeX} 
%\setmainfont{Linux Libertine O}
%\setsansfont{Mentor Std}
%\setmonofont{LMMono12-Regular}

\setmainfont{Mentor Sans Std}
%\setmainfont{Ubuntu}
%\setmathfont{XITS Math}
%\setmonofont{Ubuntu Mono}


\usepackage{fancyhdr}
\pagestyle{fancy}
\lhead{}
\chead{}
\rhead{}
\lfoot{\hspace*{-0.4in}{\tiny Luther Tychonievich}}
\cfoot{}
\rfoot{{\tiny \thepage\ of \pageref{lastpage}}\hspace*{-0.4in}}
\renewcommand{\headrulewidth}{0pt}
\renewcommand{\footrulewidth}{0pt}




\newenvironment{slide*}[1][]{
	\def\inset{\leftmargin=0.5ex}
	\clearpage
	\ifthenelse{\isempty{#1}}{}{\vspace*{-0.75in}\begin{center}\raggedright{\bf #1\textcolor{none}{j}\hrule}\vspace{0.5ex}\end{center}}
	\def\nest{\begin{list}{$^{_\bullet}$ }{\inset\def\inset{\leftmargin=1.5ex}\itemindent=0ex}}
	\def\unnest{\end{list}}
}{
}
\newenvironment{slide}[1][]{
	\begin{slide*}[#1]\nest
}{
	\unnest\end{slide*}
}

\newlength\nsbase
\setlength\nsbase{0.06\paperheight}
\def\Large{\fontsize{1.44\nsbase}{1.728\nsbase}\selectfont}
\def\large{\fontsize{1.2\nsbase}{1.44\nsbase}\selectfont}
\def\normalsize{\fontsize{\nsbase}{1.2\nsbase}\selectfont}
\def\small{\fontsize{0.8333\nsbase}{\nsbase}\selectfont}
\def\scriptsize{\fontsize{0.6944\nsbase}{0.8333\nsbase}\selectfont}
\def\tiny{\fontsize{0.5788\nsbase}{0.6944\nsbase}\selectfont}


\begin{document}%\boldmath
\raggedright


~\vfill
\begin{center}
{\large \bf FHMWG Tech Discussion 1}
\vfill
{\tiny Dr.\ Luther Tychonievich\\FHMWG Board\\University of Virginia\\~}
\end{center}
~\vfill


\begin{slide}[Outline]
\item Stage 1 goals
\item Stage 1 process
\item Common threads
\item Specific recommendations
\item Open questions
\end{slide}



\begin{slide}[Stage 1 Goals]
\item Use existing metadata standards
\item Document core metadata fields
\item Clarify use in Family History space
\end{slide}

\begin{slide*}[Stage 1 Summary]

\centering
\begin{tabular}{l@{\hspace{1em}}l}
\textbf{Topic} & \textbf{Recommendation} \\\hline
Album   & Name \& ID of albums image is in  \\
Caption & Free-text catch-all           \\
Date    & Datetime of depicted scene    \\
Event   & Free-text description         \\
Location& GPS and free-text             \\
Person  & Name and image region         \\
Object  & Title and image region        \\
\end{tabular}

\nest
\item Sources: option suited to genealogy not found
\item Rights: IPTC just text, PLUS too complex
\unnest
\end{slide*}

\begin{slide}[Stage 1 Process]
\item Board identifies key areas
\item Teams review options, draft recommendations
\item \textbf{Technical feedback}
\item Revised recommendation
\item Stakeholder feedback
\item Release
\end{slide}


\begin{slide}[Common Threads]
\item Piecemeal support OK
    \nest
    \item \dots\ as long as unsupported parts not removed
    \unnest
\item One place to write
    \nest
    \item Described in detail
    \item If present, read this too
    \unnest
\item Optional places to read
    \nest
    \item Mentioned with links
    \item Can be used as fallbacks
    \unnest
\item Write XMP
    \nest
    \item \dots\ each team's conclusion; IPTC IIM, EXIF also reviewed
    \unnest
\end{slide}


\begin{slide}[Albums]
\item From {\tt metadataworkinggroup.com} -- now defunct
\item Each image has {\scriptsize \tt mwg-coll:Collections} containing any number of albums it belongs to
\item Each album has a name and/or URI
    %\nest
    %\item A {\scriptsize \tt mwg-coll:Collections} with an {\scriptsize \tt rdf:Bag} of resources, each with a {\scriptsize \tt mwg-coll:CollectionName} and/or {\scriptsize \tt mwg-coll:CollectionURI}
    %\unnest
\item Not stage 1:
    \nest
    \item Ordinal in album, album nesting, album metadata
    \unnest
%\item Also read:
    %\nest
    %\item None
    %\unnest
\end{slide}

\begin{slide}[Caption]
\item From {\tt dublincore.org}, endorsed by XMP and IPTC
\item Each image has a {\scriptsize \tt dc:description}
\item Not stage 1:
    \nest
    \item Text markup, links to other metadata, caption metadata
    \unnest
%\item Also read:
    %\nest
    %\item 
    %EXIF~ImageDescription~(0x10E); 
    %EXIF~UserComment~(0x9286); 
    %{\scriptsize \tt photoshop:Headline};
    %{\scriptsize \tt dc:title};
    %file~name~of~image~file;
    %{\scriptsize \tt photoshop:title}
    %\unnest
\end{slide}

\begin{slide}[Date]
\item From {\tt http://ns.adobe.com/photoshop/}, endorsed by IPTC
\item Each image has a {\scriptsize \tt photoshop:DateCreated} with date of the depicted scene
\item Not stage 1:
    \nest
    \item Approximate dates, date intervals, dates BCE
    \item Dates of any other image-related event
    \unnest
%\item Also read:
    %\nest \small
    %\item 
    %IPTC~2:55~and~2:60; 
    %EXIF~DateTimeOriginal~(0x9003); 
    %{\scriptsize \tt dc:date};
    %EXIF~DateTimeDigitized~(0x9004); 
    %IPTC~2:62~and~2:63; 
    %{\scriptsize \tt xmp:CreateDate}
    %\item almost 100 more known...
    %\unnest
\end{slide}

\begin{slide}[Event]
\item From IPTC Extension schema
\item Each image has a {\scriptsize \tt Iptc4xmpExt:Event} with an description
\item Not stage 1:
    \nest
    \item Event metadata
    \unnest
%\item Also read:
    %\nest \small
    %\item none
    %\unnest
\end{slide}

\begin{slide}[Location]
\item From IPTC Extension schema, referencing EXIF
\item Each image has a {\scriptsize \tt Iptc4xmpExt:LocationShown} containing any number of locations
\item Each location has a GPS coordinate and/or name and/or URI
    %\nest
    %\item A {\scriptsize \tt Iptc4xmpExt:LocationShown} with an {\scriptsize \tt rdf:Bag} of resources, each with a {\scriptsize \tt exif:GPSLatitude}, {\scriptsize \tt exif:GPSLongitude}, and/or {\scriptsize \tt Iptc4xmpExt:LocationName}%  {\scriptsize \tt Iptc4xmpExt:LocationId}
    %\unnest
\item Not stage 1:
    \nest
    \item Jurisdiction hierarchy (City/State etc inadequate)
    \item GPS regions and/or precision
    \unnest
%\item Also read:
    %\nest \small
    %\item Other IPTC and Exif addresses and coordiantes (many)
    %\unnest
\end{slide}

\begin{slide}[Objects and People]
\item Decision: use IPTC 2019.1 image regions
\item Limit complexity:
    \nest
    \item Require regions ({\small\tt Iptc4xmpExt:ImageRegion})
    \item One person or object per region
        \nest
        \item {\small\tt Iptc4xmpExt:PersonInImageWDetails} with name, description, and/or URI
        \item {\small\tt Iptc4xmpExt:ArtworkOrObject} with title
        \unnest
    \item Use ``whole-image'' region if region unknown
    \unnest

%\end{slide}\begin{slide}[Objects and People]

%\item Each region has either a {\small\tt Iptc4xmpExt:PersonInImageWDetails} or a {\small\tt Iptc4xmpExt:ArtworkOrObject}.
    %\nest
    %\item A {\small\tt Iptc4xmpExt:PersonInImageWDetails} contains
        %\nest
        %\item a {\small\tt Iptc4xmpExt:PersonName} containing an AltLang;
        %\item optionally a {\small\tt Iptc4xmpExt:PersonDescription}
    %and/or {\small\tt Iptc4xmpExt:PersonId}
        %\unnest
    %\item A {\small\tt Iptc4xmpExt:ArtworkOrObject} contains
        %\nest
        %\item A {\small\tt Iptc4xmpExt:AOTitle} containing an AltLang
        %\unnest
    %\unnest

\end{slide}\begin{slide}[Objects and People]

\item Not stage 1:
    \nest
    \item Other person metadata
    \item Recommendation re non-person non-objects, like animals
    \item Distinction between ``unknown region'' and ``relevant but not depicted''
    \unnest
%\item Also read:
    %\nest \small
    %\item {\small\tt dc:creator}, IPTC IIM 2:80 By-Line, {\small\tt plus:ImageCreator}, {\small\tt Iptc4xmpExt:PersonInImage}, EXIF 0x13B Artist
    %\item Person, object metadata outside of any region
    %\unnest
\end{slide}


\begin{slide}[Outline]
\item Stage 1 goals
\item Stage 1 process
\item Common threads
\item Specific recommendations
\item \textbf{Open questions}
\end{slide}

\begin{slide}[Use of URI]
\item Albums, Locations, and People admit URI
    \nest
    \item Unique identifier to allow matching
    \item URL to allow linking to DB
    \unnest
\item Should we recommend using these?
    \nest
    \item Pro: more functionality
    \item Con: possibility of malicious links
        \nest
        \item Possible: sanitize to hash-based UUID
            \nest
            \item e.g., {\tt\small https://www.familysearch.org/tree/person/LC32-HZ6}
            becomes {\tt\small urn:uuid:da3fac6c-a2e5-59d7-936b-1124707fd3ef}
            \unnest
        \unnest
    \unnest
\end{slide}


\begin{slide}[One or Several?]
\item Can tools handle one image with
    \nest
    \item multiple image regions
    \item multiple people per region
    \item multiple locations
    \item multiple albums
    \item multiple languages (see next slide)
    \unnest
\item If some cannot, should we recommend
    \nest
    \item store only one
    \item first/last takes precedence
    \item custom ``primary" attribute
    \unnest
\end{slide}

\begin{slide}[Handling AltLang]
\item IPTC text fields have AltLang values:
    \nest
    \item Multiple (language, text) pairs; e.g. \\{\scriptsize\begin{verbatim}<rdf:Alt>
    <rdf:li xml:lang='en'>Gordon Clarke</rdf:li>
    <rdf:li xml:lang='ja-Latn'>Kurāku Gōdon</rdf:li>
</rdf:Alt>\end{verbatim}}
    \unnest
\item Should we recommend
    \nest
    \item Full support for AltLang
    \item Handling one language per field
    \item Handling one language per image file
    \unnest
\end{slide}


\begin{slide}[Free-form vs canonical RDF]
\item XMP is encoded as RDF/XML
    \nest
    \item can represent arbitrary property graph
    \item many ways to serialize same data
    \unnest
\item All common XMP vocabularies are trees
    \nest
    \item Tree-structured RDF's XML has nested-element normalized form
    \unnest
\item Do we say ``use nested form'' or ``use RDF toolchain''?
    \nest
    \item Nested: much less code to implement
    \item RDF: guaranteed to read any tool's output
    \unnest
\end{slide}

\begin{slide}[How much should we standardize?]
\item Handling secondary fields
    \nest
    \item ``shall read these other fields''
    \item ``shall keep these 2 fields in sync''
    \item ``if main field missing, default to this other''
        \nest
        \item ``\dots with this conversion algorithm''
        \unnest
    \unnest
\item UI guidelines
    \nest
    \item ``call it this''
        \nest
        \item ``\dots in each of these human languages''
        \unnest
    \item ``present it in this order''
    \unnest
\item Techniques for censuring/sanitizing/dropping metadata
\end{slide}

\begin{slide*}[Stage 1 Summary]

\centering
\begin{tabular}{l@{\hspace{1em}}l}
\textbf{Topic} & \textbf{Recommendation} \\\hline
Album   & Name \& ID of albums image is in  \\
Caption & Free-text catch-all           \\
Date    & Datetime of depicted scene    \\
Event   & Free-text description         \\
Location& GPS and free-text             \\
Person  & Name and image region         \\
Object  & Title and image region        \\
\end{tabular}

\nest
\item Stage 2: Sources, Rights
\item Questions: URI, Multiplicity, AltLang, RDF, secondary fields, UI/UX, sanitizing
\unnest
\end{slide*}


\label{lastpage}
\end{document}
